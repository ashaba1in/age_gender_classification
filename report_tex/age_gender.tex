Эта часть выполнена Алексеем Биршертом. \\
\subsection*{Описание метода}
В качестве базовой модели для распознавания пола и возраста используется глубокая нейронная сеть ResNet-18, последний полносвязный слой которой заменён на два полносвязных слоя с нелинейностью ReLU и дропаутом между ними.
Последний слой содержит 2 выходных нейрона для классификации по полу (мужчина или женщина) и 101 для классификации по возрасту (от 0 до 100 лет включительно).
На вход подаются фотографии лиц людей, детектированные с помощью модели распознавания лиц, размером 227 на 227 пикселей, 3 канала цвета - R, G, B\@.
Предсказанный пол определяется с помощью определения выходного нейрона соответсвующей нейронной сети с максимальным значением.
Предсказанный возраст определяется с помощью преобразования вектора значений выходных нейронов соответствующей нейронной сети функцией софтмакс.
Каждый значение вектора преобразуется в экспоненту в его степени, затем делится на сумму элементов преобразованного вектора.
После вышеописанного преобразования мы получаем вектор вероятностей, которые модель выдала для каждого возраста в диапазоне от 0 до 100 включительно.
Дальнейший подсчет точного предсказанного возраста получается с помощью подсчета матожидания возраста при соответствующем вероятностном распределении - каждый элемент вектора после применения софтмакса умножается на соответствующий ему возраст.

\subsection*{Описание данных}
В качестве датасета для обучения двух вышеописанных моделей были избраны датасеты IMDB-WIKI-101 и FGNET\@.
Распределение реального возраста в датасете IMDB-WIKI-101 имеет вид нормальной кривой со средним около 35 лет, имея значительно малое количество объектов с возрастом меньше 10 лет или больше 90.
Для восполнения данных по возрасту до 10 лет был избран датасет FGNET, в котором большая часть объектов это дети до 15 лет.
Для улучшения сходимости нейронных сетей была произведена предобработка всех объектов - в итоговую выборку не были включены объекты следующие объекты: объекты с плохо различимыми лицами (показатель уверенности модели распознавания лиц в том, что это лицо ниже фиксированного значения), объекты с некорректно заполненными данными про пол/возраст, объекты со слишком маленькими фотографиями.
Итого было получено около 200 тысяч объектов, которые были в дальнейшем поделены с сохранением баланса классов 1 к 19 на валидационную и обучающую выборки соответственно.
В качестве датасета для тестирования был избран датасет Adience, по которому известно большое количество результатов различных моделей.
Из него были исключены объекты с некорректным описанием пола или возраста.
Итого было получено почти 11 тысяч объектов для тестовой выборки.
В Adience метки возраста в формате 8 групп - 0: [0, 2], 1: [4, 6], 2: [8, 12], 3: [15, 20], 4: [25, 32], 5: [38, 43], 6: [48, 53], 7: [60, 100].
В связи с этим, необходимо было решить как относить метки реального возраста от 0 до 100 к этим группам.
Было принято решение относить к ближайшей группе - например 22 года ближе к 20, чем к 25, следовательно группа 3.
В случае одинакового расстояния выбиралась первая по порядку группа.

\subsection*{Эксперименты}
Первым необходимо было решить вопрос архитектуры базовой модели, было принято решение остановиться на ResNet-18.
# TODO: описание экспериментов.
Далее нужно было решить вопрос с разделением количеством моделей - использовать одну модель с двумя выходами или две модели, каждая с одним выходом.
# TODO: описание почему выбрал две и сравнение.
Процесс подбора гиперпараметров: # TODO: описать и графики доделать.
Процесс выбора аугментации, ну тут просто про то, какую взял, ибо я особо не экспериментировал # TODO
Процесс выбора метода обучения возраста и так далее # TODO

\subsection*{Результаты}
Сравнение на Adience с другими статьями, показать, что моя модель предсказывает хорошо, что плохо, графики # TODO