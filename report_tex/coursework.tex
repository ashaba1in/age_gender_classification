\documentclass[a4paper,14pt]{extarticle}
\usepackage{geometry}
\usepackage[T1]{fontenc}
\usepackage[utf8]{inputenc}
\usepackage[english,russian]{babel}
\usepackage{amsmath}
\usepackage{amsthm}
\usepackage{amssymb}
\usepackage{fancyhdr}
\usepackage{setspace}
\usepackage{graphicx}
\usepackage{colortbl}
\usepackage{tikz}
\usepackage{pgf}
\usepackage{subcaption}
\usepackage{listings}
\usepackage[colorlinks, linkcolor=blue, urlcolor=blue]{hyperref}
\usepackage{indentfirst}
\graphicspath{{images/}}%путь к рисункам

\makeatletter
\renewcommand{\@biblabel}[1]{#1.} % Заменяем библиографию с квадратных скобок на точку:
\makeatother

\geometry{left=2.5cm}% левое поле
\geometry{right=1.5cm}% правое поле
\geometry{top=1.5cm}% верхнее поле
\geometry{bottom=1.5cm}% нижнее поле
\renewcommand{\baselinestretch}{1.5} % междустрочный интервал

\newcommand{\bibref}[3]{\hyperlink{#1}{#2 (#3)}} % biblabel, authors, year

\renewcommand{\theenumi}{\arabic{enumi}}% Меняем везде перечисления на цифра.цифра
\renewcommand{\labelenumi}{\arabic{enumi}}% Меняем везде перечисления на цифра.цифра
\renewcommand{\theenumii}{.\arabic{enumii}}% Меняем везде перечисления на цифра.цифра
\renewcommand{\labelenumii}{\arabic{enumi}.\arabic{enumii}.}% Меняем везде перечисления на цифра.цифра
\renewcommand{\theenumiii}{.\arabic{enumiii}}% Меняем везде перечисления на цифра.цифра
\renewcommand{\labelenumiii}{\arabic{enumi}.\arabic{enumii}.\arabic{enumiii}.}% Меняем везде перечисления на цифра.цифра


\begin{document}
    \begin{titlepage}
    \newpage

    {\setstretch{1.0}
    \begin{center}
        Федеральное государственное автономное образовательное учреждение высшего образования «Национальный исследовательский университет «Высшая школа экономики»
        \\
        \bigskip
        Факультет компьютерных наук \\
        Основная образовательная программа \\
        Прикладная математика и информатика \\
    \end{center}
    }

    \vspace{8em}

    \begin{center}
    {\Large ГРУППОВАЯ КУРСОВАЯ РАБОТА}
        \\
        \textsc{\textbf{
        Программный проект на тему
        \linebreak
        "Решения на основе компьютерного зрения для проектов в городской среде"}}
    \end{center}

    \vspace{2em}

    {\setstretch{1.0}
    \hfill\parbox{16cm}{
    \hspace*{5cm}\hspace*{-5cm}Выполнили студенты группы 171, 3 курса,\\
    Биршерт Алексей Дмитриевич,\\
    Шабалин Александр Михайлович\\

    \hspace*{5cm}\hspace*{-5cm}Руководитель КР: старший преподаватель\\ Соколов Евгений Андреевич
    \\

    %\hspace*{5cm}\hspace*{-5cm}Куратор:\hfill < степень>, <звание>, <ФИО полностью>\\

    \hspace*{5cm}\hspace*{-5cm}Куратор: Магистр, разработчик решений компьютерного зрения \\
    Григорий Петрович Черномордик
    \\
    }
    }

    \vspace{\fill}

    \begin{center}
        Москва 2020
    \end{center}

\end{titlepage}
    \newpage

    {
        \hypersetup{linkcolor=black}
        \tableofcontents
    }

    \newpage

    \begin{abstract}
        Автоматическое предсказание пола и возраста человека по фотографиям, полученным в самых разных условиях -
        это важная и сложная задача, находящая применение во многих областях жизнедеятельности людей.
        В своем проекте мы стремились воплотить подход к решению этой задачи, основанный на сверточных нейронных сетях.
        Свое решение мы разбили на две составных части - детектирование лица человека и ключевых точек на его лице
        и дальнейшее предсказание пола и возраста по признакам лица.
        \par Для детектирования мы используем архитектуру RetinaFace с ResNet-18 в качестве основой модели.
        В качестве обучающих данных мы используем датасет WIDER FACE с добавленными к нему пятью ключевыми точками для каждого лица.
        На тестовой выборке наш детектор получает значение метрики Precision равное 83\%.
        \par Для классификации найденных лиц мы используем модель на основе двух ResNet-18.
        В качестве обучающих данных мы испольуем датасеты IMDB-WIKI-101 и FGNET\@.
        В качестве тестовых данных мы используем датасет Adience, наша модель получает для возраста значение метрики
        Accuracy 45\% и значение метрики One-off-accuracy 80\%, для пола значение метрики Accuracy 85\%.
        \par В итоге мы получили рабочую модель, способную обрабатывать большие массивы фотографий,
        детектировать на них людей и предсказывать для их пол и возраст в автоматическом режиме.
        Ссылка на гитхаб с проектом - \url{https://github.com/birshert/age_gender_classification}.
        \\
        \\
        \small \textbf{\textit{Ключевые слова---}}Определение возраста и пола, Детектирование лиц, Компьютерное зрение, Глубокое обучение \\
        Automatically predicting real age and gender from face images acquired in unconstrained conditions
        is an important and challenging task in many real-world applications.
        In our project we intended to construct an approach to predicting a person's real age and gender from photograph
        based on convolutional neural networks.
        Our solution is divided into solving two separate subtasks - detecting one's face and it's landmarks and
        further age and gender estimation based on facial features.
        \par For face detection we use RetinaFace architecture with ResNet-18 as a backbone.
        For training we use WIDER FACE dataset with added five facial landmarks for every face.
        On testing subset we achieve 83\% Precision result.
        \par For real age and gender estimation we use a model based on two ResNet-18.
        For training we use IMDB-WIKI-101 dataset and FGNET dataset.
        For testing we use Adience dataset, our model achieves 45\% Accuracy, 80\% One-off-accuracy for age estimation
        and 85\% Accuracy for gender prediction.
        \par As a result, we got a working model capable of processing large volumes of photos,
        detecting people faces on them and predicting their real age and gender in automatic mode.
        Github project link - \url{https://github.com/birshert/age_gender_classification}.
        \\
        \small \textbf{\textit{Keywords---}}Age and gender estimation, Facial detection, Computer vision, Deep learning
        \\
        \newpage
    \end{abstract}

    \section{Введение}\label{sec:введение}
    \subsection{Описание предметной области}\label{subsec:описание-предметной-области}
Во многих сферах деятельности данные такого рода как пол и возраст человека имеют большое значение.
Антропология, маркетинг, социология - во всех этих областях наличие инструмента, позволяющего быстро получать такие данные, очень ценно.
Хорошим источником для получения этих данных могут служить фотографии людей.
Однако исследование большого объема фотографий вручную занимает слишком много времени,
за которое данные могут устареть и перестать быть актуальными,
к тому же ручная разметка довольно дорого стоит.
В своей работе мы стремимся решить задачу автоматизации получения данных о возрасте и поле человека по его фотографии.
Данная задача относится к задачам компьютерного зрения.
\par Самые ранние подходы к классификации пола и возраста опирались на известные зависимости размеров и форм черт лица,
так называемые антропометрические модели.
Более поздние подходы начали использовать сверточные нейронные сети.
Все методы так или иначе используют в первую очередь изображение лица человека в качестве основного признакового описания.
В зависимости от преследуемой цели инженеры ставят перед собой следующие задачи: улучшить качество предсказаний,
увеличить скорость работы и научиться работать с фотографиями плохого качества.

\subsection{Постановка задачи}\label{subsec:постановка-задачи}
\par Проблему предсказания возраста и пола нужно рассматривать в разделении на подзадачи - детекция и выравнивание лица,
определение пола и определение возраста.
\par Наша цель в задаче детекции и выравнивания лица заключается в наилучшем определении ключевых точек на лице,
соответствующих глазам, и наилучшем определении ограничивающего контура, обрамляющем лицо на фотографии.
При этом для нас гораздо предпочтительнее не найти какое-нибудь лицо, чем выделить объект, не являющийся лицом.
По этой причине мы будем стараться максимизировать метрику $Precision$:
\[Precision = \frac{|\text{detected faces}|}{|\text{all detected objects}|}\]
Мы будем считать, что найденный объект является лицом, если модель уверена в нем больше, чем на 50\%.
Также стоит заметить, что правильность выставления ключевых точек для нас тоже не критична,
поэтому функционал ошибки мы зададим как \[L = L_{cls} + 0.25 L_{box} + 0.1 L_{pts},\]
где $L_{cls}$ - ошибка классификации, $L_{box}$ - ошибка контура и $L_{pts}$ - ошибка ключевых точек.
\par Наша цель в задаче определения пола заключается в достижении наилучшего качества бинарной классификации.
\[\sum\limits_{i = 1}^{n} \left[ \mathcal{G}(x_i) = y_i \right] \to \max\limits_{\mathcal{G}}, \]
где $x$ - объекты, $y$ - их пол, $\mathcal{G}$ - модель, предсказывающая пол.
\par В задаче определения возраста мы будем минимализировать разницу между предсказанным возрастом и реальным возрастом человека.
\[\sum\limits_{i=1}^{n} \left\| \mathcal{A}(x_i) - y_i \right\| \to \min\limits_{\mathcal{A}}, \]
где $x$ - объекты, $y$ - их возраст, $\mathcal{A}$ - модель, предсказывающая возраст.
\par Итогом проектной работы должна стать программно реализованная система, принимающая на вход от пользователя каталог фотографий
и возвращающая пол и возраст людей, запечатленных на фотографиях.
В первую очередь мы будем стремиться получить наилучшее качество классификации, но при этом не будем забывать про время работы.
Система должна опираться на использование сверточных нейронных сетей.
\par Дальнейшая работа описана в следующих главах: Обзор литературы, Детектирование лиц,
Классификация гендерных и возрастных групп, Описание системы для пользователя, Заключение.
\par "Детектирование лиц"\, выполнено Александром Шабалиным, "Классификация гендерных и возрастных групп"\, Алексеем Биршертом.


    \section{Обзор литературы}\label{sec:обзор-литературы}
    \subsection{Детектирование лиц}\label{subsec:детектирование-лиц}
Детектирование лиц на фотографии - одна из древнейших задач компьютерного зрения, возникшая в 1990-х годах.
Первые хорошие результаты появились в 2004 году.
Описанный в статье~\cite{face_detection} метод находил лица с помощью признаков Хаара, используя каскад детекторов, обученных алгоритмом AdaBoost.
В 2014 году в статье~\cite{face_detection2} был предложен метод, использующий Deformable Parts AgeGender (DPM).
Его идея заключается в нахождении зависимостей между подвижными частями.
Например, лицо представляется, как нечто, состоящее из глаз, носа, рта, расположенных в некотором антропоморфическом виде.
Однако все описанные методы показывали довольно плохие результаты в сложных случаях, так как опирались на ограниченный, придуманный людьми набор признаков.
Поэтому методы, основанные на сверточных нейронных сетях быстро вытеснили остальные.
Лучшие известные на данный момент подходы описаны в статье~\cite{face_detection3}.
Все из них используют нейронные сети (обычно ResNet~\cite{resnet}) для получения признаков.
Так как модель сама находит признаки и зависимости, результат получается лучше.
\par Немаловажной задачей является и выравнивание лиц.
Самый быстрый метод получить выровненное лицо - определить ключевые точки на лице и преобразовать изображение так, чтобы эти точки были на заранее определенных местах.
В статье~\cite{align} описан метод вычисления координат основных точек на лице человека - окаймляющих лицо, глаза, нос, рот и брови.
Вычисление точек происходит с помощью каскада регрессоров, обучаемых с помощью градиентного бустинга.
\par В нашей работе мы пользуемся архитектурой RetinaFace~\cite{retinaface}.
На данный момент она является state-of-the-art в задаче детектирования лиц и показывает впечатляющие результаты.
Одна из ее отличительных черт - умение предсказывать пять ключевых точек, необходимых для выравнивания изображения.

\subsection{Классификация возраста и пола}\label{subsec:классификация-возраста-и-пола}
\par Задачи определения пола и возраста человека находят применение в разных сферах жизни человека, в наружном наблюдении, в антропологии, в биометрической идентификации.
Современные подходы к этой задаче опираются на сверточные нейронные сети.
Так, например, в статье~\cite{hassner} описано решение с помощью сверточной сети небольшой глубины.
Для классификации возраста и пола используется одна и та же архитектура.
Нейронная сеть состоит из трёх свёрточных слоёв и двух полносвязных,
небольшой размер сети объясняется желанием быть физичным в распознавании лиц и нежеланием переобучиться.
Точность по классификации пола была 86.8 $\pm$ 1.4\%,
возрастных групп - 50.7 $\pm$ 5.1\% для точного попадания в группу и 84.7 $\pm$ 2.2\% для попадания в правильную или соседнюю для датасета Adience.
\par В статье~\cite{INDIA} описан алгоритм анализа лица с помощью пяти сверточных нейросетей, получающих изображения лица целиком, левого и правого глаза, носа и рта соответственно.
Итоговое решение принимается на основе выходов всех пяти нейросетей.
Нейросеть, получающая на вход всё изображение лица, имеет три сверточных слоя, прочие по два.
Точность по классификации пола достигла 89.6 $\pm$ 1.3\%,
возрастных групп - 54.3 $\pm$ 3.5\% для точного попадания в группу и 87.6 $\pm$ 1.9\% для попадания в правильную или соседнюю на датасете Adience,
что является значительным улучшением результата предыдущей статьи.
\par В статье~\ref{imdb} описан метод использования ансамбля глубоких сверточных нейронных сетей для предсказания реального возраста.
Для этого авторы статьи собрали свой собственный датасет IMDB-WIKI-101,
дообучали на нём модели с архитектурой VGG-16 предобученные на ImageNet-1000,
потом классифицировали признаки, полученные сверточной нейронной сетью с помощью двух полносвязных слоев
и использовали выход второго слоя для предсказания возраста от 0 до 100 включительно.
В процессе они использовали различные улучшения в выравнивании лиц (вращали изображение вдоль центра
и выбирали детектированное лицо с максимальным показателем уверенности модели детектирования лиц),
затем тренировали ансамбль из 20 моделей на датасете LAP~\ref{LAP} для получения оптимального результата.
В итоге им удалось получить результат MAE по возрасту 3.2 на датасете IMDB-WIKI-101.
\newpage


    \section{Детектирование лиц}\label{sec:детектирование-лиц}
    Эта часть сделана Александром Шабалиным

\subsection{Описание выбранного метода}\label{subsec:описание-выбранного-метода}
Для детектирования лиц была использована архитектура нейронной сети RetinaFace.
Она состоит из двух частей: основной модели и пирамиды признаков.
Основная модель - это сверточная нейронная сеть, предназначенная для выявления признаков из изображения.
В качестве основной модели может выступать MobileNet, VGG, ResNet и другие.
\par Идея пирамиды признаков довольно проста.
При детектировании лиц мы хотим находить лица разных размеров.
Для этого строятся две пирамиды из пяти карт признаков в каждой.
Первая получается проходом снизу в верх, а вторая - сверху вниз.
Слоями первой пирамиды являются выходы четырех слоев основной модели такие,
что размер каждого следующего выхода в два раза меньше размера предыдущего.
Так мы получаем карты разного размера, что позволяет детектировать как маленькие, так и большие лица.
Вторая пирамида необходима из-за того, что на ранних слоях сверточной сети содержится значительно меньше семантической информации,
а значит, предсказания лиц на ранних слоях менее точны и мы должны как-то компенсировать это.
Слои второй пирамиды мы получаем, проходя сверху вниз следующим образом.
Первый (верхний) слой получается с помощью наложения свертки с ядром размера 1х1 на верхний слой первой пирамиды.
Каждый следующий из пяти слоев получается путем поэлементого суммирования предыдущего слоя,
увеличенного в два раза, со сверткой размера 1х1 соответствующего слоя первой пирамиды.
Таким образом нам удается передать большим по размеру картам признаков семантическую информацию меньших карт, компенсировав ее нехватку.
Предсказания получаются с помощью применения нескольких сверточных слоев к картам признаков второй пирамиды и конкатенированием результатов этих сверток.
\\
\par В своей реализации в качестве основной модели используется ResNet18 предобученный на ImageNet.
Выходы четырех блоков ResNet являются четыремя картами признаков первой пирамиды,
пятая карта признаков получается наложением свертки с ядром размера 3х3 и шагом 2 на последнюю карту признаков.
Так как датасет ImageNet содержит в себе картинки с различными изображениями,
а в нашей задаче необходимо находить лица людей, мы замораживем только первый слой ResNet, а остальные дообучаем.
Выбор ResNet с 18 слоями обосновывается тем, что в нашей задаче очень важна скорость работы.
С увеличением размера сверточной сети скорость ее работы заметно падает, а качество увеличивается не так сильно.
% TODO Это будет видно на графиках
% TODO написать о pyramid feature

\subsection{Подготовка данных и обучение}\label{subsec:подготовка-данных-и-обучение}
Для обучения модели был использован датасет WIDER FACE\@.
Он содержит 32,203 фотографии с 393,703 лицами на них.
Для каждого лица хранятся координаты пяти ключевый точек: две для глаз, одна для носа и две для рта,
а также координаты прямоугольной рамки, ограничивающей лицо.
В качестве аугментации мы пробовали делать случайный переворот изображения,
однако он не улучшил результаты, поэтому мы решили от него отказаться.
Входные изображения масштабируются до квадратных следующим образом: большая сторона становится равной 256,
а к меньшей добавляется нулевой отступ с двух сторон так, чтобы размер итогового изображения составлял 256 на 256.
При обучении был использован Adam оптимизатор со скоростью обучения $10^{-3}$.
Модель обучается в течение 20-ти эпох с размером батча 32.
\par Ошибка модели считается по формуле: \[L = L_{cls}(p, p^*) + 0.25 L_{box}(t, t^*) + 0.1 L_{pts}(l, l^*),\]
где $p$ - вероятность лица, $p^*$ - истинный ответ (0 или 1),
для подсчета ошибки классификатора $L_{cls}(p, p^*)$ используется софтмакс ошибка для бинарных классов классов (лицо/не лицо).\\
$t = \{t_1, t_2, t_3, t_4\}$ и $t^* = \{t^*_1, t^*_2, t^*_3, t^*_4\}$ предсказанные и истинные координаты ограничивающей рамки соответственно.
Для подсчета ошибки $L_{box}(t, t^*)$ используется функция $\text{Smooth-L}_1$ loss.
\\$l = \{l_{x1}, l_{y1}, \dots, l_{x5}, l_{y5}\}$ и $l = \{l^*_{x1}, l^*_{y1}, \dots, l^*_{x5}, l^*_{y5}\}$ предсказанные и истинные координаты ключевых точек лица.
Для подсчета ошибки $L_{pts}(l, l^*)$ так же используется $\text{Smooth-L}_1$ loss.

\subsection{Выравнивание}\label{subsec:выравнивание}
Перед передачей найденных лиц классификатору возраста и пола необходимо их выровнить.
Выравнивание производится на основе двух ключевых точек для глаз, найденных вместе с лицами.
Для этого находится угол отклонения прямой, проведенной с помощью точек глаз, от горизонтали.
После этого мы домножаем матрицу изображения на матрицу поворота таким образом, что линия глаз становится горизонтальной.
Мы не используем остальные точки, так как они избыточны, а в некоторых случаях даже мешают.
Например, расположение точек рта у улыбающегося человека и у серьезного отличаются, но это отличие не должно никак влиять на выравнивание.
Глаза же всегда располагаются на одном месте относительно лица, что позволяет их считать хорошей опорой для выравнивания.

\subsection{Эксперименты}\label{subsec:эксперименты2}
Для определения с выбором основной модели были проведены сравнения между ResNet18, ResNet50, VGG и MobileNet.
По результатам этого эксперимента выяснилось, что ResNet50 получает лучший результат,
но для предсказания ответа ему требуется гораздо больше времени, чем ResNet18.
MobileNet работает быстрее остальных, но ошибка такой модели оказывается больше остальных.
VGG показывает хороший результат, но работает медленнее всех.
В результате выбор остановился на ResNet18, потому что баланс скорости и качества детектирования лиц этой модели мы считаем оптимальным.
Графики для этого эксперимента появятся чуть позже, потому что они не успели посчитаться :(

% TODO график сравнения с ResNet другого размера, VGG и MobileNet. Сказать, что другие реснеты гораздо медленее, что для нас критично.
Для ускорения времени работы алгоритма мы пробовали уменьшать число карт признаков в пирамиде признаков,
однако в таком случае некоторые слишком большие или слишком маленькие лица переставали обнаруживаться,
что крайне негативно отражалось на результатах модели.
% TODO посмотреть, что будет при изменении числа слоев в feature pyramid, попробовать менять параметры якорей.
% TODO попробовать другой оптимизатор.
% TODO оценить качество предсказания landmarks и показать, что другие методы работают хуже.

\subsection{Результаты}\label{subsec:результаты2}
Полученная модель хорошо справляется с поставленной задачей, получая значение метрики Precision равное 83\%.
Для нашей задачи эта метрика имеет крайне важное значение, так как ты не должны классифицировать несуществующих людей.
Несмотря на то, что реализация авторов RetinaFace получала 91\%, мы считаем это хорошим результатом,
так как из-за нехватики вычислительных мощностей пришлось почти в 3 раза снизить размер стороны входного изображения,
использовать ResNet18 вместо ResNet152, а также уменьшить количество эпох обучения, что заметно сказывается на результатах работы.

\newpage

\subsection{Примеры работы модели}\label{subsec:примеры-работы-модели}
\begin{figure}[h!]
    \centering
    \begin{subfigure}[b]{1.02\linewidth}
        \includegraphics[width=\linewidth]{images/good1.jpg}
        \caption{Модель смогла найти 50 лиц из 61 указанного в ответе.}
    \end{subfigure}

    \centering
    \begin{subfigure}[b]{1.02\linewidth}
        \includegraphics[width=\linewidth]{images/norm1.jpg}
        \caption{Модель хорошо справилась со своей задачей, найдя все повернутые к камере лица, однако она также выделила пустую часть стола.}
    \end{subfigure}
\end{figure}



    \section{Классификация гендерных и возрастных групп}\label{sec:классификация-гендерных-и-возрастных-групп}
    Эта часть выполнена Алексеем Биршертом. \\
\subsection*{Описание метода}
В качестве базовой модели для распознавания пола и возраста используется глубокая нейронная сеть ResNet-18, последний полносвязный слой которой заменён на два полносвязных слоя с нелинейностью ReLU и дропаутом между ними.
Последний слой содержит 2 выходных нейрона для классификации по полу (мужчина или женщина) и 101 для классификации по возрасту (от 0 до 100 лет включительно).
На вход подаются фотографии лиц людей, детектированные с помощью модели распознавания лиц, размером 227 на 227 пикселей, 3 канала цвета - R, G, B\@.
Предсказанный пол определяется с помощью определения выходного нейрона соответсвующей нейронной сети с максимальным значением.
Предсказанный возраст определяется с помощью преобразования вектора значений выходных нейронов соответствующей нейронной сети функцией софтмакс.
Каждый значение вектора преобразуется в экспоненту в его степени, затем делится на сумму элементов преобразованного вектора.
После вышеописанного преобразования мы получаем вектор вероятностей, которые модель выдала для каждого возраста в диапазоне от 0 до 100 включительно.
Дальнейший подсчет точного предсказанного возраста получается с помощью подсчета матожидания возраста при соответствующем вероятностном распределении - каждый элемент вектора после применения софтмакса умножается на соответствующий ему возраст.

\subsection*{Описание данных}
В качестве датасета для обучения двух вышеописанных моделей были избраны датасеты IMDB-WIKI-101 и FGNET\@.
Распределение реального возраста в датасете IMDB-WIKI-101 имеет вид нормальной кривой со средним около 35 лет, имея значительно малое количество объектов с возрастом меньше 10 лет или больше 90.
Для восполнения данных по возрасту до 10 лет был избран датасет FGNET, в котором большая часть объектов это дети до 15 лет.
Для улучшения сходимости нейронных сетей была произведена предобработка всех объектов - в итоговую выборку не были включены объекты следующие объекты: объекты с плохо различимыми лицами (показатель уверенности модели распознавания лиц в том, что это лицо ниже фиксированного значения), объекты с некорректно заполненными данными про пол/возраст, объекты со слишком маленькими фотографиями.
Итого было получено около 200 тысяч объектов, которые были в дальнейшем поделены с сохранением баланса классов 1 к 19 на валидационную и обучающую выборки соответственно.
В качестве датасета для тестирования был избран датасет Adience, по которому известно большое количество результатов различных моделей.
Из него были исключены объекты с некорректным описанием пола или возраста.
Итого было получено почти 11 тысяч объектов для тестовой выборки.
В Adience метки возраста в формате 8 групп - 0: [0, 2], 1: [4, 6], 2: [8, 12], 3: [15, 20], 4: [25, 32], 5: [38, 43], 6: [48, 53], 7: [60, 100].
В связи с этим, необходимо было решить как относить метки реального возраста от 0 до 100 к этим группам.
Было принято решение относить к ближайшей группе - например 22 года ближе к 20, чем к 25, следовательно группа 3.
В случае одинакового расстояния выбиралась первая по порядку группа.

\subsection*{Эксперименты}
Первым необходимо было решить вопрос архитектуры базовой модели, было принято решение остановиться на ResNet-18.
# TODO: описание экспериментов.
Далее нужно было решить вопрос с разделением количеством моделей - использовать одну модель с двумя выходами или две модели, каждая с одним выходом.
# TODO: описание почему выбрал две и сравнение.
Процесс подбора гиперпараметров: # TODO: описать и графики доделать.
Процесс выбора аугментации, ну тут просто про то, какую взял, ибо я особо не экспериментировал # TODO
Процесс выбора метода обучения возраста и так далее # TODO

\subsection*{Результаты}
Сравнение на Adience с другими статьями, показать, что моя модель предсказывает хорошо, что плохо, графики # TODO
    \newpage


    \section{Описание системы для пользователя}\label{sec:описание-системы-для-пользователя}
    Система написана и протестирована на операционной системе Ubuntu 18.04 LTS и Ubuntu 20.04 LTS с ЦПУ Intel Core i7 и ГПУ Nvidia RTX 2080Ti\@.
Для оптимальной скорости работы необходимо не менее 8Гб ОЗУ и 6Гб видеопамяти ГПУ.
\par Система для пользователя представлена в виде скрипта на языке программирования Python3.
Для использования нашей программы, пользователь должен скачать файл \textbf{install.sh} из репозитория с сайта github.com по следующей ссылке~\url{https://github.com/birshert/age_gender_classification}.
После скачивания скрипт установит все необходимые пакеты и подготовит виртуальное пространство для работы.
Для работы с программой необходимо выгрузить фотографии в директорию и подать путь к директории в качестве аргумента командной строки скрипту \textbf{workflow.py}.
По завершению работы в текущей директории появится .csv файл с таблицей со следующими колонками: имя файла, удалось ли найти лицо на фотографии,
количество найденных лиц, массив с предсказанными возрастами, массив с предсказанными полами.
    \newpage


    \section{Заключение}\label{sec:выводы}
    Решая поставленную задачу о классификации людей на фотографиях по полу и возрасту мы получили алгоритм,
    позволяющий быстро и качественно находить лица людей на фотографиях, а затем предсказывать их пол и возраст.
    При решении мы опробовали множество различных подходов и остановились на архитектуре RetinaFace для детектирования лиц
    и двух ResNet для классификации пола и возраста.
    \par Наша модель для детекции лиц показывает значение метрики Precision равное 83\% на датасете WIDER FACE~\cite{wider}.
    Наша модель для классификации пола и возраста показывает значение метрики Accuracy 45\% и
    значение метрики One-off-accuracy 80\%, для пола значение метрики Accuracy 85\% на датасете Adience~\cite{adience}.
    \par Для улучшения качества классификации в дальнейшем у нас есть несколько идей, которые мы не успели реализовать.
    Определенно положительно бы сказалась дальнейшая работа по улучшению качества всех моделей, по уменьшению их размера и скорости работы.
    Также имеет смысл добавить в структуру нашего решения фильтрацию неживых лиц - памятников и лиц на рекламных щитах и постерах.
    \newpage


    \section{Список литературы}\label{sec:список-литературы}
    \begin{thebibliography}{0}

    \bibitem{resnet}\hypertarget{resnet}{}
    \href{https://arxiv.org/abs/1512.03385}
    {
        Deep Residual Learning for Image Recognition.
        Kaiming He, Xiangyu Zhang, Shaoqing Ren, Jian Sun.
        In IEEE 2015
    }

    \bibitem{vgg}\hypertarget{vgg}{}
    \href{https://arxiv.org/pdf/1409.1556.pdf}
    {
        Very deep convolutional networks for large-scale image recognition.
        Karen Simonyan, Andrew Zisserman.
        In ICLR 2015
    }

    \bibitem{face_detection}\hypertarget{face_detection}{}
    \href{http://www.face-rec.org/algorithms/Boosting-Ensemble/16981346.pdf}
    {
        Robust Real-Time Face Detection.
        Paul Viola, Michael J. Jones.
        In IEEE 2003
    }

    \bibitem{face_detection2}\hypertarget{face_detection2}{}
    \href{http://rodrigob.github.io/documents/2014_eccv_face_detection_with_supplementary_material.pdf}
    {
        Face detection without bells and whistles.
        Makrus Mathias, Rodrigo Beneson, Marco Pedersoli, Lus Van Gool.
        In ECCV 2014
    }

    \bibitem{face_detection3}\hypertarget{face_detection3}{}
    \href{https://arxiv.org/abs/1905.01585}
    {
        Accurate Face Detection for High Performance.
        Faen Zhang, Xinyu Fan, Guo Ai, Jianfei Song, Yongqiang Qin, Jiahong Wu.
        In ArXiv 2019
    }

    \bibitem{align}\hypertarget{align}{}
    \href{http://www.csc.kth.se/~vahidk/papers/KazemiCVPR14.pdf}
    {
        One Millisecond Face Alignment with an Ensemble of Regression Trees.
        Vahid Kazemi and Josephine Sullivan.
        In IEEE 2014
    }

    \bibitem{retinaface}\hypertarget{retinaface}{}
    \href{https://arxiv.org/abs/1905.00641}
    {
        RetinaFace: Single-stage Dense Face Localisation in the Wild.
        Jiankang Deng, Jia Guo, Yuxiang Zhou, Jinke Yu, Irene Kotsia, Stefanos Zafeiriou.
        In ArXiv 2019
    }

    \bibitem{age1994}\hypertarget{age1994}{}
    \href{https://pdfs.semanticscholar.org/20cb/d360c8e6f70aac3e11853d81e3b18e4866c2.pdf}
    {
        Age Classification from Facial Images.
        Young H. Kwon, Niels da Vitoria Lobo.
        In IEEE 1994
    }


    \bibitem{hassner}\hypertarget{hassner}{}
    \href{https://talhassner.github.io/home/projects/cnn_agegender/CVPR2015_CNN_AgeGenderEstimation.pdf}
    {
        Age and Gender Classification using Convolutional Neural Networks.
        Gil Levi, Tal Hassner.
        In IEEE 2015
    }

    \bibitem{adience}\hypertarget{adience}{}
    \href{https://talhassner.github.io/home/projects/Adience/Adience-data.html}
    {
        Adience Database
    }

    \bibitem{imdb}\hypertarget{imdb}{}
    \href{https://data.vision.ee.ethz.ch/cvl/publications/papers/proceedings/eth_biwi_01229.pdf}
    {
        DEX: Deep EXpectation of apparent age from a single image.
        Rasmus Rothe, Radu Timofte, Luc Van Gool.
        In IJCV 2016
    }

    \bibitem{imdb_db}\hypertarget{imdb_db}{}
    \href{https://data.vision.ee.ethz.ch/cvl/rrothe/imdb-wiki/}
    {
        IMDB-WIKI-101 Database
    }

    \bibitem{lap}\hypertarget{lap}{}
    \href{http://refbase.cvc.uab.es/files/EFP2015.pdf}
    {
        ChaLearn Looking at People 2015: apparent age and cultural event recognition datasets and results.
        Sergio Escalera, Junior Fabian, Pablo Pardo, Xavier Baro, Jordi Gonzalez, Hugo J. Escalante, Marc Oliu, Dusan Misevic, Ulrich Steiner, Isabelle Guyon.
        In EFP 2015
    }

    \bibitem{ror}\hypertarget{ror}{}
    \href{https://arxiv.org/pdf/1710.02985.pdf}
    {
        Age group and gender estimation in the wild with deep RoR architecture.
        Ke Zhang, Ce Gao, Liru Guo, Miao Sun, Xingfang Yuan, Tony X. Han, Zhenbing Zhao and Baogang Li.
        In IEEE 2017
    }

    \bibitem{lstm}\hypertarget{lstm}{}
    \href{https://arxiv.org/pdf/1805.10445.pdf}
    {
        Fine-grained age estimation in the wild with attention LSTM networks.
        Ke Zhang, Na Liu, Xingfang Yuan, Xinyao Guo, Ce Gao, Zhenbing Zhao and Zhanyu Ma.
        In ArXiv 2018
    }

    \bibitem{order}\hypertarget{order}{}
    \href{https://arxiv.org/pdf/1901.07884.pdf}
    {
        Rank-consistent ordinal regression for neural networks.
        Wenzhi Cao, Vahid Mirjalili, Sebastian Raschka.
        In ArXiv 2019
    }

    \bibitem{morph}\hypertarget{morph}{}
    \href{https://ebill.uncw.edu/C20231_ustores/web/classic/store_main.jsp?STOREID=4}
    {
        MORPH-2 Database
    }

    \bibitem{utk}\hypertarget{utk}{}
    \href{https://susanqq.github.io/UTKFace/}
    {
        UTKFace Database
    }

    \bibitem{afad}\hypertarget{afad}{}
    \href{https://afad-dataset.github.io/}
    {
        AFAD Database
    }

    \bibitem{mobile}\hypertarget{mobile}{}
    \href{https://arxiv.org/abs/1704.04861}
    {
        MobileNets: Efficient Convolutional Neural Networks for Mobile Vision Applications.
        Andrew G. Howard, Menglong Zhu, Bo Chen, Dmitry Kalenichenko, Weijun Wang, Tobias Weyand, Marco Andreetto, Hartwig Adam.
        In ArXiv 2017
    }

    \bibitem{imagenet}\hypertarget{imagenet}{}
    \href{http://www.image-net.org/}
    {
        ImageNet-1000 Database
    }

    \bibitem{wider}\hypertarget{wider}{}
    \href{http://shuoyang1213.me/WIDERFACE/}
    {
        WIDER-FACE Database
    }

    \bibitem{shuffle}\hypertarget{shuffle}{}
    \href{https://arxiv.org/abs/1807.11164}
    {
        ShuffleNet V2: Practical Guidelines for Efficient CNN Architecture Design.
        Ningning Ma, Xiangyu Zhang, Hai-Tao Zheng, Jian Sun.
        In ECCV 2018
    }

    \bibitem{fgnet}\hypertarget{fgnet}{}
    \href{https://yanweifu.github.io/FG_NET_data/}
    {
        FGNET Database
    }

\end{thebibliography}

\end{document}