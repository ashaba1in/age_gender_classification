\subsection{Описание предметной области}\label{subsec:описание-предметной-области}
Во многих сферах деятельности данные такого рода как пол и возраст человека имеют большое значение.
Антропология, маркетинг, социология - во всех этих областях наличие инструмента, позволяющего быстро получать такие данные, очень ценно.
Хорошим источником для получения этих данных могут служить фотографии людей.
Однако исследование большого объема фотографий вручную занимает слишком много времени,
за которое данные могут устареть и перестать быть актуальными,
к тому же ручная разметка довольно дорого стоит.
В своей работе мы стремимся решить задачу автоматизации получения данных о возрасте и поле человека по его фотографии.
Данная задача относится к задачам компьютерного зрения.
\par Самые ранние подходы к классификации пола и возраста опирались на известные зависимости размеров и форм черт лица,
так называемые антропометрические модели.
Более поздние подходы начали использовать сверточные нейронные сети.
Все методы так или иначе используют в первую очередь изображение лица человека в качестве основного признакового описания.
В зависимости от преследуемой цели инженеры ставят перед собой следующие задачи: улучшить качество предсказаний,
увеличить скорость работы и научиться работать с фотографиями плохого качества.

\subsection{Постановка задачи}\label{subsec:постановка-задачи}
\par Проблему предсказания возраста и пола нужно рассматривать в разделении на подзадачи - детекция и выравнивание лица,
определение пола и определение возраста.
\par Наша цель в задаче детекции и выравнивания лица заключается в наилучшем определении ключевых точек на лице,
соответствующих глазам, и наилучшем определении ограничивающего контура, обрамляющем лицо на фотографии.
При этом для нас гораздо предпочтительнее не найти какое-нибудь лицо, чем выделить объект, не являющийся лицом.
По этой причине мы будем стараться максимизировать метрику $Precision$:
\[Precision = \frac{|\text{detected faces}|}{|\text{all detected objects}|}\]
Мы будем считать, что найденный объект является лицом, если модель уверена в нем больше, чем на 50\%.
Также стоит заметить, что правильность выставления ключевых точек для нас тоже не критична,
поэтому функционал ошибки мы зададим как \[L = L_{cls} + 0.25 L_{box} + 0.1 L_{pts},\]
где $L_{cls}$ - ошибка классификации, $L_{box}$ - ошибка контура и $L_{pts}$ - ошибка ключевых точек.
\par Наша цель в задаче определения пола заключается в достижении наилучшего качества бинарной классификации.
\[\sum\limits_{i = 1}^{n} \left[ \mathcal{G}(x_i) = y_i \right] \to \max\limits_{\mathcal{G}}, \]
где $x$ - объекты, $y$ - их пол, $\mathcal{G}$ - модель, предсказывающая пол.
\par В задаче определения возраста мы будем минимализировать разницу между предсказанным возрастом и реальным возрастом человека.
\[\sum\limits_{i=1}^{n} \left\| \mathcal{A}(x_i) - y_i \right\| \to \min\limits_{\mathcal{A}}, \]
где $x$ - объекты, $y$ - их возраст, $\mathcal{A}$ - модель, предсказывающая возраст.
\par Итогом проектной работы должна стать программно реализованная система, принимающая на вход от пользователя каталог фотографий
и возвращающая пол и возраст людей, запечатленных на фотографиях.
В первую очередь мы будем стремиться получить наилучшее качество классификации, но при этом не будем забывать про время работы.
Система должна опираться на использование сверточных нейронных сетей.
\par Дальнейшая работа описана в следующих главах: Обзор литературы, Детектирование лиц,
Классификация гендерных и возрастных групп, Описание системы для пользователя, Заключение.
\par "Детектирование лиц"\, выполнено Александром Шабалиным, "Классификация гендерных и возрастных групп"\, Алексеем Биршертом.