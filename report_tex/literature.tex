\subsection{Детектирование лиц}\label{subsec:детектирование-лиц}
Детектирование лиц на фотографии - одна из древнейших задач компьютерного зрения, возникшая в 1990-х годах.
Первые хорошие результаты появились в 2004 году.
Описанный в статье~\cite{face_detection} метод находил лица с помощью признаков Хаара, используя каскад детекторов, обученных алгоритмом AdaBoost.
В 2014 году в статье~\cite{face_detection2} был предложен метод, использующий Deformable Parts AgeGender (DPM).
Его идея заключается в нахождении зависимостей между подвижными частями.
Например, лицо представляется, как нечто, состоящее из глаз, носа, рта, расположенных в некотором антропоморфическом виде.
Однако все описанные методы показывали довольно плохие результаты в сложных случаях, так как опирались на ограниченный, придуманный людьми набор признаков.
Поэтому методы, основанные на сверточных нейронных сетях быстро вытеснили остальные.
Лучшие известные на данный момент подходы описаны в статье~\cite{face_detection3}.
Все из них используют нейронные сети (обычно ResNet~\cite{resnet}) для получения признаков.
Так как модель сама находит признаки и зависимости, результат получается лучше.
\par Немаловажной задачей является и выравнивание лиц.
Самый быстрый метод получить выровненное лицо - определить ключевые точки на лице и преобразовать изображение так, чтобы эти точки были на заранее определенных местах.
В статье~\cite{align} описан метод вычисления координат основных точек на лице человека - окаймляющих лицо, глаза, нос, рот и брови.
Вычисление точек происходит с помощью каскада регрессоров, обучаемых с помощью градиентного бустинга.
\par В нашей работе мы пользуемся архитектурой RetinaFace~\cite{retinaface}.
На данный момент она является state-of-the-art в задаче детектирования лиц и показывает впечатляющие результаты.
Одна из ее отличительных черт - умение предсказывать пять ключевых точек, необходимых для выравнивания изображения.

\subsection{Классификация возраста и пола}\label{subsec:классификация-возраста-и-пола}
\par Задачи определения пола и возраста человека находят применение в разных сферах жизни человека, в наружном наблюдении, в антропологии, в биометрической идентификации.
Современные подходы к этой задаче опираются на сверточные нейронные сети.
Так, например, в статье~\cite{hassner} описано решение с помощью сверточной сети небольшой глубины.
Для классификации возраста и пола используется одна и та же архитектура.
Нейронная сеть состоит из трёх свёрточных слоёв и двух полносвязных,
небольшой размер сети объясняется желанием быть физичным в распознавании лиц и нежеланием переобучиться.
Точность по классификации пола была 86.8 $\pm$ 1.4\%,
возрастных групп - 50.7 $\pm$ 5.1\% для точного попадания в группу и 84.7 $\pm$ 2.2\% для попадания в правильную или соседнюю для датасета Adience.
\par В статье~\cite{INDIA} описан алгоритм анализа лица с помощью пяти сверточных нейросетей, получающих изображения лица целиком, левого и правого глаза, носа и рта соответственно.
Итоговое решение принимается на основе выходов всех пяти нейросетей.
Нейросеть, получающая на вход всё изображение лица, имеет три сверточных слоя, прочие по два.
Точность по классификации пола достигла 89.6 $\pm$ 1.3\%,
возрастных групп - 54.3 $\pm$ 3.5\% для точного попадания в группу и 87.6 $\pm$ 1.9\% для попадания в правильную или соседнюю на датасете Adience,
что является значительным улучшением результата предыдущей статьи.
\par В статье~\ref{imdb} описан метод использования ансамбля глубоких сверточных нейронных сетей для предсказания реального возраста.
Для этого авторы статьи собрали свой собственный датасет IMDB-WIKI-101,
дообучали на нём модели с архитектурой VGG-16 предобученные на ImageNet-1000,
потом классифицировали признаки, полученные сверточной нейронной сетью с помощью двух полносвязных слоев
и использовали выход второго слоя для предсказания возраста от 0 до 100 включительно.
В процессе они использовали различные улучшения в выравнивании лиц (вращали изображение вдоль центра
и выбирали детектированное лицо с максимальным показателем уверенности модели детектирования лиц),
затем тренировали ансамбль из 20 моделей на датасете LAP~\ref{LAP} для получения оптимального результата.
В итоге им удалось получить результат MAE по возрасту 3.2 на датасете IMDB-WIKI-101.
\newpage