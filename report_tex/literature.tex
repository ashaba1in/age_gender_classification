\subsection{Архитектура ResNet}\label{subsec:архитектура-resnet}
Так как при детекции и классификации лиц мы пользуемся архитектурой сверточной сети ResNet, мы считаем нужным пояснить, как она устроена.
\par ResNet~\cite{resnet} создавалась по подобию VGG~\cite{vgg}.
Так же, как у VGG большинство ее сверток используют размер ядра 3x3, при проходе в глубину размер карты признаков уменьшается, а количество каналов увеличивается.
Используемая в нашем случае версия ResNet-18 имеет следующую архитектуру.
Первый слой является сверткой с ядром размера 7x7 и сдвигом 2, после которого идет 3x3 maxpooling с шагом 2.
Затем располагаются 4 блока в каждом блоке содержится по 4 сверточных слоя с ядром 3х3 с функцией активации ReLU\@.
Количество фильтров сверточных слоев в первом блоке равно 64, с каждым следующим блоком число фильтров удваивается.
Во всех блоках, кроме первого, первый сверточный слои имеет шаг 2, из-за чего размер карты признаков сокращается вдвое.
Таким образом с каждым новым все блоки требуют одинаковое количество операций с плавующей точкой и поэтому их выполнение занимает одинаковое количество времени.
В конце располагаются слой average pool и полносвязный слой.
\par Отличительной чертой ResNet являются быстрые соединения.
Их смысл состоит в передачи информации одного слоя не только следующему прямо после него, но и следующему за ним через два.
Такой подход предотвращает затухание градиента и позволяет модели продолжать обучаться.
Быстрые соединения повторяются в ResNet через каждые два слоя.

\subsection{Детектирование лиц}\label{subsec:детектирование-лиц}
Детектирование лиц на фотографии - одна из древнейших задач компьютерного зрения, возникшая в 1990-х годах.
Первые хорошие результаты появились в 2004 году.
Описанный в статье~\cite{face_detection} метод находил лица с помощью признаков Хаара, используя каскад детекторов, обученных алгоритмом AdaBoost.
В 2014 году в статье~\cite{face_detection2} был предложен метод, использующий Deformable Parts AgeGender (DPM).
Его идея заключается в нахождении зависимостей между подвижными частями.
Например, лицо представляется, как нечто, состоящее из глаз, носа, рта, расположенных в некотором антропоморфическом виде.
Однако все описанные методы показывали довольно плохие результаты в сложных случаях, так как опирались на ограниченный, придуманный людьми набор признаков.
Поэтому методы, основанные на сверточных нейронных сетях быстро вытеснили остальные.
Лучшие известные на данный момент подходы описаны в статье~\cite{face_detection3}.
Все из них используют нейронные сети (обычно ResNet) для получения признаков.
Так как модель сама находит признаки и зависимости, результат получается лучше.
\par Немаловажной задачей является и выравнивание лиц.
Самый быстрый метод получить выровненное лицо - определить ключевые точки на лице и преобразовать изображение так, чтобы эти точки были на заранее определенных местах.
В статье~\cite{align} описан метод вычисления координат основных точек на лице человека - окаймляющих лицо, глаза, нос, рот и брови.
Вычисление точек происходит с помощью каскада регрессоров, обучаемых с помощью градиентного бустинга.
\par В нашей работе мы пользуемся архитектурой RetinaFace~\cite{retinaface}.
На данный момент она является state-of-the-art в задаче детектирования лиц и показывает впечатляющие результаты.
Одна из ее отличительных черт - умение предсказывать пять ключевых точек, необходимых для выравнивания изображения.

\subsection{Классификация возраста и пола}\label{subsec:классификация-возраста-и-пола}
\par В прошлом, такие задачи решались с помощью алгоритмов, построенных на основе исследования фиксированных, выделенных людьми, признаков,
но это не приносило удовлетворительных результатов.
Так происходило потому что люди, запечатленные на фотографиях в неформальных условиях,
имеют слишком большую вариацию поз, размеров относительно размера кадра, освещения и так далее.
\par Все известные подходы к предсказанию возраста и пола человека по фото опираются на исследование его лица.
Самые ранние подходы к определению возраста людей по фотографии
использовали в качестве ключевого фактора различия в пропорциях и размерах черт лица в зависимости от возраста и пола -
антропометрические модели~\cite{age1994}.
Все они вычисляли положение ключевых точек на лице - глаза, нос, рот, скулы - и анализировали их в дальнейшем.
\par Более поздние подходы начали применять сверточные нейронные сети.
Сверточные сети используются в этой задаче в силу своей способности к извлечению признаков из изображений.
А как мы упомянули ранее, именно невозможность сконструировать признаки служила препятствием для качественного анализа.
Так, в статье~\cite{hassner} описано использование сверточных нейронных сетей небольшой глубины -
три сверточных слоя и два полносвязных.
В статье описана работа с датасетом Adience~\cite{adience}, в котором возраст представлен в виде 8 возрастных групп.
Таким образом модель училась предсказывать наиболее вероятный диапазон возраста.
У авторов статьи удалось добиться результата в $50.7\% \pm 5.1 \%$ Accuracy, $84.7\% \pm 2.2\%$ One-off-accuracy для предсказания возрастной группы,
$86.8\% \pm 1.4 \%$ Accuracy для классификации пола на Adience benchmark, где One-off-accuracy это метрика,
равная доле ответов, которые попали либо в правильную возрастную группу, либо в соседнюю.
\par С развитием области компьютерного зрения и появлением новых, более сложных, позволяющих достигать всё лучшие результаты,
архитектур сверточных нейронных сетей, на основе этих архитектур появляются и решения для задачи предсказания возраста и пола.
В более современных статьях, например~\cite{imdb}, используется глубокая сверточная нейронная сеть архитектуры VGG-16,
что позволяет улучшить качество по сравнению с предшествующими решениями.
Стоит отметить, что в этой статье описано предсказание реального возраста от 0 до 100 лет включительно, каждая категория это целое число от 0 до 100.
То есть модель училась предсказывать более точный возраст, нежели диапазон.
В этой статье описан процесс сбора и подготовки данных для обучения на датасете IMDB-WIKI-101\cite{imdb_db}.
В результате авторы статьи добиваются результата MAE в $3.22$ в конкурсе~\cite{lap}.
\par В статье~\cite{ror} используется архитектура RoR, Residual networks of Residual networks,
которая представляет собой модификацию моделей ResNet.
Авторы статьи достигают качества в $67.3\% \pm 3.6 \%$ Accuracy,
$97.5\% \pm 0.7\%$ One-off-accuracy для предсказания возрастной группы на Adience benchmark.
\par Хочется отметить наличие статей, использующих архитектуры на основе ResNet или RoR с LSTM (архитектура реккурентной нейронной сети)~\cite{lstm}.
В этой работе сверточные нейронные сети используются для выделения глобальных признаков, отвечающих за возраст,
а реккурентные нейронные сети используются для выделения и работы с локальными признаками, что в итоге позволяет добиться впечатляющих результатов.
Авторы приводят результат в $67.8\% \pm 3 \%$ Accuracy,
$97.6\% \pm 0.6\%$ One-off-accuracy для предсказания возрастной группы на Adience benchmark.
\par Стоит обратить внимание на взгляд на задачу предсказания возраста как на задачу классификации с определенным на классах отношением порядка.
Например, в статье \cite{order} описывается процесс обучения сверточных нейронных сетей для предсказания пола по фотографии человека.
В статье авторы приводят алгоритм, на основе которого строится этот подход - каждый выходной нейрон соответствует задаче бинарной классификации вида "больше числа".
Предложенный авторами подход позволяет легко предсказывать итоговый возраст как количество выходных нейронов, значения которых после сигмоидного преобразования дали больше 0.5.
В статье приведено доказательство этого метода и описаны эксперименты.
Эта модель является SOTA на трёх датасетах, использующихся для сравнения моделей предсказания возраста - MORPH-2 \cite{morph}, UTKFace \cite{utk} и AFAD \cite{afad}.
\newpage